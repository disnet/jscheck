\documentclass{article}
\usepackage{listings}

\usepackage{float}
 
\floatstyle{ruled}
\newfloat{program}{thp}{lop}
\floatname{program}{Snippet}


\begin{document}

\title{Project Report - Limited Static Type Checking for JavaScript}
\author{Huascar A. Sanchez \and Tim Disney}

\maketitle

\lstset{showstringspaces=false}

\section{Introduction}
It is well known that in the construction of software it is generally 
cheaper to catch programming errors earlier in the development cycle \cite{cc2}.
Static type checking is one method for catching errors during compile time
but many languages such as JavaScript are dynamically typed and so discovery
of certain type of errors is deferred until runtime.

Our project was an implementation of a static type checker for JavaScript which
checked properties on object's prototype. We call our implementation JSCheck.

Our implementation is both unsound and incomplete. One thing that sets our 
implementation apart for previous work is that our checker works (in a limited fashion) for
full JavaScript.

\section{Implementation}
JSCheck was implemented in Haskell. Used the HJS library to provide full
parsing of JavaScript (ECMAScript 3rd edition plus some additions from JavaScript 1.5)

We extract the types and check them on methods that have type annotations.
\begin{program}
\begin{verbatim}
function Dog() {
  //..
}

Dog.prototype.getName = function(){
  //..
}

Dog.prototype.bark = function(){
  //..
}

//# @tyep Dog fido
function useDog(fido) {
  fido.bark();
}

var d = new Dog();
useDog(d);
\end{verbatim}
\caption{Type Checking}
\end{program}

\section{Related Work}

Tom and Caitlin did this \cite{fwjsStruct}.

The $JS_0$ guy did this \cite{typeinferenceforjavascriptEcoop} and this \cite{typecheckingforjavascript}.

The Ruby guy did this \cite{typecheckingruby}.

Google has the closure compiler \cite{closureCompiler}.

\section{Conclusion}

\bibliographystyle{abbrv}
\bibliography{report}

\end{document}


